%
\documentclass[runningheads]{llncs}

\usepackage{graphicx}
\usepackage{url}


\begin{document}
%
\title{Modellierung und Simulation 2019/2020 Conway's Game of Life}
\author{Louis Donath, Dario Klepoch}
%
\authorrunning{}
\institute{Potsdam University}


\maketitle              

\begin{abstract}
Conway's Game of Life ist ein Automat...
Hier kommt eine kurze Zusammenfassung des Projektes rein
%\keywords{First keyword  \and Second keyword \and Another keyword.}
\end{abstract}
%

\begin{section}{Einfuehrung}
\end{section}

\section{aasd}
    

\begin{section}{Implementierung}
    
    Bei der Betrachtung der Implementierung werden wir ueber die Implementierung der Randbedingungen sprechen.
    Weil wir mit unseren ersten Implementierung mit einer unzureichenden Performance hatten,
    werden wir im zweiten Teil ueber Performanceverbesserungen sprechen.
   
    

    \subsection{Randbedingungen}
        In unserer Implementierung des GoL haben wir drei unterschiedliche Randbedingungen implementiert.
        Eine Randbedingung sagt aus wie sich das Spiel verhaelt, wenn auf Zellen ausserhalb des Spielbrettes zugegriffen wird.
        
        \subsubsection{Absorbierende} Randbedingung

    \subsection{Performance}

    \subsubsection{Nachbarn ermitteln}

    \subsubsection{Partielle Updates}
        Nach der ersten Implementierung des GoL war die Performance 

    \subsubsection{Numpyarray anstatt Pythonlist}
        Um die Zeit 

    
    \subsubsection{Weitere Performancesteigerung}
        sind durch sehr viele unterschiedliche Veraenderungen moeglich.
        Eine sehr grossen Performancesteigerung ist dadurch moeglich einen effizientere Datenstruktur als ein (numpy-)array zu verwenden.
        In verschiedenen anderen Implementierungen des GoL wird hierfuer ein Quadtree benutzt.
        Ein Quadtree wird meistens dafuer verwendet effizient 2-dimensionale Daten zu speichern \cite{quadtreeGeeksForGeeks}.
        Da das GoL auch 2D-Daten sind ist es ein perfekter Anwendungsbereich fuer einen Quadtree.
        Mit `Haslife' wurde das GoL auf diese Weise implementiert \cite{haslifeWiki}. \newline
        Um die Performance noch weiter zu steigern ist es moeglich den Quadtree parallel aufzubauen.
        Hier kann entweder die CPU oder auch die GPU benutzt werden.
        In \cite{quadtreesOnGPU} wurden lineare Quadtrees verwendet um einen Quadtree vollstaendig auf der GPU aufzubauen.

\end{section}


\begin{thebibliography}{8}
\bibitem{quadtreeGeeksForGeeks}
    \url{https://www.geeksforgeeks.org/quad-tree/}, letzter Zugriff: 27.3.2020

\bibitem{haslifeWiki}
    \url{https://en.wikipedia.org/wiki/Hashlife}, letzter Zugriff: 27.3.2020

\bibitem{quadtreesOnGPU}
    Dupuy, Jonathan \& Iehl, Jean-Claude \& Poulin, Pierre. (2018).
    Quadtrees on the GPU. 10.1201/9781351052108-12. 

\end{thebibliography}
\end{document}
